\documentclass{article}

\begin{document}

\title{How to optimize Linux on Asus E200HA}
\author{matteo bonanomi}

% % % % % % % % % % % % % % % % % % % % % % % % % % % %
% TITLE, ABSTRACT AND TABLE OF CONTENTS
% % % % % % % % % % % % % % % % % % % % % % % % % % % %

\maketitle

\begin{abstract}
The present guide aims at providing general guidelines for the optimization of Linux distros on Asus E200HA. It is not an easy task, since this netbook has not a Linux-friendly hardware (Intel Cherrytrail... looking at you!), yet it can be a decent machine for school/university purposes. You can expect a relevant improvement in system performance and much more available disk space compared to Windows 10. Everythig I learned in six months of daily usage will be listed in this document. Feel free to share these pieces of information, contact me via GitHub and ask for improvements. Enjoy!  
\end{abstract}

\tableofcontents

\newpage

% % % % % % % % % % % % % % % % % % % % % % % % % % % %
% DISTRO SUGGESTIONS
% % % % % % % % % % % % % % % % % % % % % % % % % % % %

\section{Which distro?}
This guide is not about hey-this-distro-is-absolutely-the-best-ever: bull***ts. There is no perfect distro, especially for this kind of laptop! It has a modest hardware, but with the right distro you can deal with it quite well. Moreover, the hardware is not Linux freindly at all... this is the most tricky part and this is what this guide is about. 

I list here some of the suggested choices for new users or intermediate Linux users that don't want to spend so much time reading posts in forums/blogs trying to find the best option for this machine. 

\subsection{SUggested distros}
\begin{itemize}
	\item Manjaro XFCE is a good option. Lightweight and intuitive thanks to XFCE Desktop Environment. Always updated thanks to the rolling Arch-based repos. For this kind of laptop, updated kernel may mean a better support for Intel processor, while a lightweight DE is necessary to guarantee good performance. The only drawback is the little but very friendly community, if you come from Ubuntu-based distro, you have to expect less documentation ready-for-use, but you can always ask for help in its community.
	\item Other Manjaro flavors. Manjaro comes with many flavors suppoerted by its great community. In this very moment, I am writing from Manjaro i3wm Community Edition. It is great, lightweight and fast... buit it is not for every user, especially if you are new to Linux. Anyway, check out Manjaro flavors, some lightweight options may be interesting.
	\item Lubuntu 18.04 and Xubuntu 18.04 are the best options for new users approaching Liinux. Especially Xubuntu is easy to manage, easy to customize and familiar to those coming from a Windows experience. Lubuntu is similar, slightly lighter but with less applications and functionalities out-of-the-box. IMHO, Xubuntu with XFCE is great!
\end{itemize}

% % % % % % % % % % % % % % % % % % % % % % % % % % % %
% HARDWARE CONFIGURATION
% % % % % % % % % % % % % % % % % % % % % % % % % % % %

\section{Hardware configuration}

\subsection{Wifi setup}
Wifi should be recognized out-of-the-box, but you need to enable third-party apps and non-free drivers when you install your Linux distro. In case you are unsure about wifi support, try your distro from a live usb!
\begin{itemize}
	\item Ubuntu and flavors. The recommended version is 18.04 or above, since proprietary wifi drivers were quite buggy in previous LTS 16.04, and totally absent in 14.04. Enable third-party app download during installation.
	\item Manjaro Linux. Enable non-free driver when using you live distro and during installation. From Manjaro 17.x, wifi is recongized out-of-the-box.
\end{itemize}

\subsubsection{Audio [EXPERIMENTAL]}

\subsubsection{eMMC Storage}
To increase performance and prolong flash memory lifespan, it is suggested to modify the following line of the configuration file \emph{/etc/fstab.} For every Linux partition, add at the fourth column the options \emph{discard,noatime}. For instance, the fourth column of the root partition should be like:
\begin{verbatim}
errors=remount-ro 0
\end{verbatim}
It must be changed in:
\begin{verbatim}
discart,noatime,errors=remount-ro 0
\end{verbatim}

\subsubsection{Multi-monitor configurations}
Asus E200HA has a very small monitor, very interesting for working on a train/bus or taking notes during lectures and conferences, yet it might me useful to configure a second monitor for home working. 

\begin{itemize}
	\item I personally tried to configure an external display connecting it to the micr-HDMI port on \textbf{Lubuntu 18.04} and \textbf{Xubuntu 18.04} and it worked perfectly out-of-the-box. Display settings GUI integrated in Ubuntu is alway a good tool to customize you multi-monitor experience without any line of code.
	\item On \textbf{Manjaro} Linux 17.x the second monitor must be configured. In case you need to do so, there is a simple GUI tool colled arandr that is the easiest way to do so. It is compatible with any Linux distro... just look for \emph{arandr} package and install it. Then run it from your app menu or from command line (just type \emph{arandr}). A simple user interface will allow you to update the list of available monitor and decide which is the primary monitor, which is the position of the secondary monitor, etc..
\end{itemize}

\paragraph{Manual configuration tips}

In case you have set you multi-monitor confiugration with arandr, please remember to save you config file and use it whwnever you want. You must open the file and run the lines of code inside it. Multimonitor configuration will be applied.

You can also configure the system to run your custom multi-monitor settings startup. I assume you have already saved the arandr configuration file with tour preferred options inside it. You can open this config file with a text editor. You will read one or more lines of code with arandr configurations. Copy all the lines. Paste all those lines at the end of the text file named \emph{.xinitrc} in your home directory. If not present, create it. Than reboot and check that the new configuration is correctly loaded. 
 
% % % % % % % % % % % % % % % % % % % % % % % % % % % %
% SOFTWARE OPTIMIZATION
% % % % % % % % % % % % % % % % % % % % % % % % % % % %

\section{Software optimization}

\subsubsection{Improve battery life}

\subsubsection{Remove (or reduce) screen tearing}

\subsubsection{Disable hybernation and suspension}


\end{document}
