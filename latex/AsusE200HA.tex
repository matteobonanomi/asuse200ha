\documentclass{article}

\begin{document}

\title{How to optimize Linux on Asus E200HA}
\author{matteo bonanomi}

% % % % % % % % % % % % % % % % % % % % % % % % % % % %
% TITLE, ABSTRACT AND TABLE OF CONTENTS
% % % % % % % % % % % % % % % % % % % % % % % % % % % %

\maketitle

\begin{abstract}
The present guide aims at providing some guidelines for the optimization of any Linux distro on Asus E200HA. It is not an easy task, since the tiny netbook has not a Linux-friendly hardware (Intel Cherrytrail... looking at you!), yet it can be a decent machine for school/university purposes. I am going to list what I have learned after about six months using this netbook for web browsing, office applications and coding coding coding. Feel free to share these pieces of information, contact me via GitHub and ask for improvements. Enjoy!
\end{abstract}

\tableofcontents

\newpage

% % % % % % % % % % % % % % % % % % % % % % % % % % % %
% DISTRO SUGGESTIONS
% % % % % % % % % % % % % % % % % % % % % % % % % % % %

\section{Which distro?}
This guide is not about hey-this-distro-is-absolutely-the best bull***ts. The is no perfect distro, especially for this kind of laptop! It has a modest hardware, hence modest performance. Moreover, the hardware is not Linux freindly at all... this is the most tricky part and this is what this guide is about. 

Nevertheless, I list here some of the suggested choices for new users or intermediete Linux users that don't want to spend so much time reading posts in forums trying to find the best option for this machine. 

\subsection{Ubuntu and Debian-based distros}

\begin{itemize}
	\item For those who are not much skilled with Linux and especially for new users, I warmly suggest an Ubuntu-based distro. In particular Lubuntu, which is the lightest official Ubuntu flavor and features a classic desktop environment like LXDE. My favourite Ubuntu flavor is Xubuntu, featureing XFCE Desktop Environment: it is not much heavier than Xubuntu and its user experience is far way more complete. This is another good option, especially for its long-term support (usually 3 years) what it comes to LTS. 
	\item In any case, choose an LTS version of Ubuntu, as updated as you can, for several good reasons. First of all an LTS is an LTS, it is a long-term supported distro, so you don't have to worry about updating it every few months. Moreover, non-LTS Ubunntu versions tend to be unreliable and unstable. Since the poor hardware of this machine is already a little bit unreliable, I can't see any good reason to worsen the actual situation. Go for an LTS, the more updated, the more chance you have to see everything working after the installation.
	\item If you want to stick with Debian, you can. Please remember that you will need proprietary driver to hope to have a reasonable user experience with this netbook, hence I don't suggest Debian and its strict non-free software management to a new user. Moreover, Debian stable tends to be quite outdated, and you will need recent Linux kenrel versions to have the most of the hardware recognized out-of-the-box. If you know what to do, I think that Debian testing or Sid with Xfce/Lxde or i3WM could be a good option, even though not for every user.
	\item I have tested Xubuntu 18.04 and Lubuntu 18.04 on Asus E200HA and I can guarantee that everything essential (except audio) works fine out of the box: graphic card, wifi card, keyboard display brightness, good battery life. 
\end{itemize}

\subsection{Arch-based distros}
\begin{itemize}
	\item Arch Linux wiki has some very good tips to improve Linux performance and reliability on out machine. Hence, of course Arch Linux is suggested! It is always updated and the first official patches for the audio recognition will arrive for sure in Arch. Nevertheless, it is not for basic users. It could be tricky to configure everything, so I assume that if you are gonna choose Arch, you know what you are doing.
	\item I think that a very good option is Manjaro Linux, in particular its official XFCE version. I am currently writing from Manjaro Linux i3WM Community Edition and it is extremely fast and reliable on this netbook. It is a interesting distro, rollis as Arch, but muuch more easier to manage since it has its own stable repository that makes life easier for new users. Everything works ot of the box, just as Ubuntu does. Like Ubuntu, audio does not work, since there is not official kernel patch to figure that out (at the moment of writing).
\end{itemize}

% % % % % % % % % % % % % % % % % % % % % % % % % % % %
% HARDWARE CONFIGURATION
% % % % % % % % % % % % % % % % % % % % % % % % % % % %

\section{Hardware configuration}

\subsection{Wifi setup}
Wifi should be recognized out-of-the-box, but you need to enable third-party apps and non-free drivers when you install your Linux distro. In case you are unsure about wifi support, try your distro from a live usb!
\begin{itemize}
	\item Ubuntu and flavors. The recommended version is 18.04 or above, since proprietary wifi drivers were quite buggy in previous LTS 16.04, and totally absent in 14.04. Enable third-party app download during installation.
	\item Manjaro Linux. Enable non-free driver when using you live distro and during installation. From Manjaro 17.x, wifi is recongized out-of-the-box.
\end{itemize}

\subsubsection{Audio [EXPERIMENTAL]}

\subsubsection{eMMC Storage}
To increase performance and prolong flash memory lifespan, it is suggested to modify the following line of the configuration file \emph{/etc/fstab.} For every Linux partition, add at the fourth column the options \emph{discard,noatime}. For instance, the fourth column of the root partition should be like:
\begin{verbatim}
errors=remount-ro 0
\end{verbatim}
It must be changed in:
\begin{verbatim}
discart,noatime,errors=remount-ro 0
\end{verbatim}

\subsubsection{Multi-monitor configurations}
Asus E200HA has a very small monitor, very interesting for working on a train/bus or taking notes during lectures and conferences, yet it might me useful to configure a second monitor for home working. 

\begin{itemize}
	\item I personally tried to configure an external display connecting it to the micr-HDMI port on \textbf{Lubuntu 18.04} and \textbf{Xubuntu 18.04} and it worked perfectly out-of-the-box. Display settings GUI integrated in Ubuntu is alway a good tool to customize you multi-monitor experience without any line of code.
	\item On \textbf{Manjaro} Linux 17.x the second monitor must be configured. In case you need to do so, there is a simple GUI tool colled arandr that is the easiest way to do so. It is compatible with any Linux distro... just look for \emph{arandr} package and install it. Then run it from your app menu or from command line (just type \emph{arandr}). A simple user interface will allow you to update the list of available monitor and decide which is the primary monitor, which is the position of the secondary monitor, etc..
\end{itemize}

\paragraph{Manual configuration tips}

In case you have set you multi-monitor confiugration with arandr, please remember to save you config file and use it whwnever you want. You must open the file and run the lines of code inside it. Multimonitor configuration will be applied.

You can also configure the system to run your custom multi-monitor settings startup. I assume you have already saved the arandr configuration file with tour preferred options inside it. You can open this config file with a text editor. You will read one or more lines of code with arandr configurations. Copy all the lines. Paste all those lines at the end of the text file named \emph{.xinitrc} in your home directory. If not present, create it. Than reboot and check that the new configuration is correctly loaded. 
 
% % % % % % % % % % % % % % % % % % % % % % % % % % % %
% SOFTWARE OPTIMIZATION
% % % % % % % % % % % % % % % % % % % % % % % % % % % %

\section{Software optimization}

\subsubsection{Improve battery life}

\subsubsection{Remove (or reduce) screen tearing}

\subsubsection{Disable hybernation and suspension}

\end{document}
